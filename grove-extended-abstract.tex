\documentclass[nonacm, acmsmall, screen, review]{acmart}

%%
%% \BibTeX command to typeset BibTeX logo in the docs
\AtBeginDocument{%
  \providecommand\BibTeX{{%
    \normalfont B\kern-0.5em{\scshape i\kern-0.25em b}\kern-0.8em\TeX}}}

\usepackage{stmaryrd}
\usepackage{mathpartir}

% Relations
% G \overset{(s,\e)}{\longrightarrow} G\left[ \e \mapsto s \sqcup G(\e) \right],
\newcommand{\action}[1]{\xrightarrow{#1}}
\newcommand{\applyAction}[3]{#1 \action{#2} #3}
\newcommand{\extends}[2]{#1\mathopen{}\left[ #2 \right]\mathclose{}}

% Functions
\newcommand{\decompOp}{\texttt{decomp}}
\newcommand{\decomp}[1]{\decompOp\mathopen{}\left(#1\right)\mathclose{}}

\newcommand{\recompOp}{\texttt{recomp}}
\newcommand{\recomp}[1]{\recompOp\mathopen{}\left(#1\right)\mathclose{}}

% Helpers
\newcommand{\edgesOp}{\text{edges}}
\newcommand{\edges}[1]{\edgesOp\mathopen{}\left(#1\right)\mathclose{}}

\newcommand{\sortOp}{\text{sort}}
\newcommand{\sort}[1]{\sortOp\mathopen{}\left(#1\right)\mathclose{}}

\newcommand{\arityOp}{\text{arity}}
\newcommand{\arity}[1]{\arityOp\mathopen{}\left(#1\right)\mathclose{}}

% Graphs
\newcommand{\e}{\varepsilon}

% Graph Edits
\newcommand{\graphEdit}[2]{\big(#1, #2\big)}

% Sets
\newcommand{\SetOf}[1]{\left\{#1\right\}}
\newcommand{\E}{\mathcal{E}}

% Sequences
\newcommand{\SequenceOf}[1]{\left(#1\right)}

% UUIDs
\newcommand{\fresh}[1]{{#1}^{+}}

% Terms
\newcommand{\id}[1]{\textcolor{gray}{\ensuremath{#1}}}
\newcommand{\eid}[2]{{#2}^{\id{#1}}}

\newcommand{\eVar}[2]{\eid{#1}{#2}}
\newcommand{\eFun}[4]{\eid{#1}{\lambda}#2{:}#3{.}#4}
\newcommand{\eApp}[3]{\eid{#1}{\left(#2~#3\right)}}
\newcommand{\eNum}[2]{\eid{#1}{\underline{#2}}}
\newcommand{\ePlus}[3]{#2~\eid{#1}{\texttt{+}}~#3}
\newcommand{\eTimes}[3]{#2~\eid{#1}{\texttt{*}}~#3}
\newcommand{\pVar}[2]{\eid{#1}{#2}}
\newcommand{\tArrow}[3]{#2 \eid{#1}{\rightarrow} #3}
\newcommand{\tNum}[1]{\eid{#1}{Num}}

\newcommand{\hole}{\ensuremath{\square}} %\textcolor{violet}{\llparenthesis}}\textcolor{violet}{\rrparenthesis}}
\newcommand{\conflictHole}[1]{%
{\noexpandarg\StrSubstitute{#1}{,}{\textcolor{red}{\,\textbf{|}\,}}[\myargs]%
{\textcolor{red}{\textbf{\{}}\myargs\textcolor{red}{\textbf{\}}}}}}%
\newcommand{\emptyHole}[2]{\square_{\id{(#1, #2)}}}
\newcommand{\multiVertex}[1]{\textcolor{red}{\ensuremath{\curlyveedownarrow_{#1}}}}
\newcommand{\cycleVertex}[1]{\textcolor{red}{\ensuremath{\circlearrowleft_{#1}}}}
\newcommand{\cursor}[1]{{\vartriangleright}#1{\vartriangleleft}}

% Constructors
\newcommand{\ExpVar}{\mathsf{Exp\_var}}
\newcommand{\ExpLam}{\mathsf{Exp\_lam}}
\newcommand{\ExpApp}{\mathsf{Exp\_app}}
\newcommand{\ExpNum}{\mathsf{Exp\_num}}
\newcommand{\ExpPlus}{\mathsf{Exp\_plus}}
\newcommand{\ExpTimes}{\mathsf{Exp\_times}}
\newcommand{\PatVar}{\mathsf{Pat\_var}}
\newcommand{\TypNum}{\mathsf{Typ\_num}}
\newcommand{\TypArrow}{\mathsf{Typ\_arrow}}

% Positions
\newcommand{\Root}{\mathsf{Root}}
\newcommand{\LamParam}{\mathsf{Param}}
\newcommand{\LamType}{\mathsf{Type}}
\newcommand{\LamBody}{\mathsf{Body}}
\newcommand{\AppFun}{\mathsf{Fun}}
\newcommand{\AppArg}{\mathsf{Arg}}
\newcommand{\PlusLeft}{\mathsf{Left}}
\newcommand{\PlusRight}{\mathsf{Right}}
\newcommand{\TimesLeft}{\mathsf{Left}}
\newcommand{\TimesRight}{\mathsf{Right}}
\newcommand{\ArrowArg}{\mathsf{Arg}}
\newcommand{\ArrowResult}{\mathsf{Result}}

% User Actions
\newcommand{\RepositionOp}{\text{Reposition}}
\newcommand{\Reposition}[2]{\RepositionOp\mathopen{}\left(#1, #2\right)\mathclose{}}

% tikz

\usepackage{tikz}
\usetikzlibrary{babel} % Ensure compatibility the 'babel' package

% Define outline versions of + and -
\def\outlinepad{0.4pt}
\def\outlinestroke{0.4pt}
\newcommand{\Plus}{\mathord{
\begin{tikzpicture}[anchor=base, baseline]
%\node at (0,0) {+};
\path[draw, line width=\outlinestroke]
   ( 0.333em+\outlinestroke/2+\outlinepad,  0.270em+\outlinestroke/2+\outlinepad)
 --( 0.333em+\outlinestroke/2+\outlinepad,  0.229em-\outlinestroke/2-\outlinepad)
 --( 0.021em+\outlinestroke/2+\outlinepad,  0.229em-\outlinestroke/2-\outlinepad)
 --( 0.021em+\outlinestroke/2+\outlinepad, -0.084em-\outlinestroke/2-\outlinepad)
 --(-0.020em-\outlinestroke/2-\outlinepad, -0.084em-\outlinestroke/2-\outlinepad)
 --(-0.020em-\outlinestroke/2-\outlinepad,  0.229em-\outlinestroke/2-\outlinepad)
 --(-0.333em-\outlinestroke/2-\outlinepad,  0.229em-\outlinestroke/2-\outlinepad)
 --(-0.333em-\outlinestroke/2-\outlinepad,  0.270em+\outlinestroke/2+\outlinepad)
 --(-0.020em-\outlinestroke/2-\outlinepad,  0.270em+\outlinestroke/2+\outlinepad)
 --(-0.020em-\outlinestroke/2-\outlinepad,  0.583em+\outlinestroke/2+\outlinepad)
 --( 0.021em+\outlinestroke/2+\outlinepad,  0.583em+\outlinestroke/2+\outlinepad)
 --( 0.021em+\outlinestroke/2+\outlinepad,  0.270em+\outlinestroke/2+\outlinepad)
 --cycle
 ;
\end{tikzpicture}
}}

\newcommand{\Minus}{\mathord{
\begin{tikzpicture}[anchor=base, baseline]
%\node at (0,0) {$-$};
\path[draw, line width=\outlinestroke]
   ( 0.306em+\outlinestroke/2+\outlinepad,  0.270em+\outlinestroke/2+\outlinepad)
 --( 0.306em+\outlinestroke/2+\outlinepad,  0.229em-\outlinestroke/2-\outlinepad)
 --(-0.306em-\outlinestroke/2-\outlinepad,  0.229em-\outlinestroke/2-\outlinepad)
 --(-0.306em-\outlinestroke/2-\outlinepad,  0.270em+\outlinestroke/2+\outlinepad)
 --cycle
 ;
\end{tikzpicture}
}}

%%%%%%%%%%%%%%%%%%%%%%%%%%%%%%%%%%%%%%%%%%%%%%%%%%%%%%%%%%%%%%%%%%%%%%%%%%%%%%%%

\begin{document}

\title[Grove]{Grove: A Convergent Collaborative Structure Editor Calculus}

% \author{Michael D. Adams}
% \orcid{0000-0003-3160-6972}

\author{Eric Griffis}
\orcid{0000-0003-1693-6172}

% \author{Cyrus Omar}
% \orcid{0000-0003-4502-7971}
% \affiliation{
%   %\position{Assistant Research Scientist}
%   \department[0]{Computer Science and Engineering}
%   \department[1]{Electrical Engineering and Computer Science}
%   \department[2]{College of Engineering}
%   \institution{University of Michigan}
%   \streetaddress{Bob and Betty Beyster Building, 2260 Hayward Street}
%   \city{Ann Arbor}
%   \state{MI}
%   \postcode{48109-2121}
%   \country{USA}
% }

\maketitle

%%%%%%%%%%%%%%%%%%%%%%%%%%%%%%%%%%%%%%%%%%%%%%%%%%%%%%%%%%%%%%%%%%%%%%%%%%%%%%%%

% TODO: replace terms with expressions

\section{Introduction}
\label{sec:introduction}

% (avoid creating a fight in the intro by claiming things the reader may summarily reject)

% (Cyrus: We use version control systems and we even use live share)
Programming is increasingly collaborative.
Programmers use a variety of tools, ranging from batch version control systems like Git~\cite{chacon_pro_2014} to real time collaboration tools such as Live Share~\cite{sole_visual_2018}, to coordinate changes to a common code base.

% (introduce structure editing)
% (Cyrus: in recent years, structure editing has been an active research area... talk about Hazel, Scratch, MPS, etc.
These collaboration tools largely assume that programs are edited as text.
However, structure editing has become increasingly prominent in both research and practice in recent years with tools like 
Scratch~\cite{resnick_scratch_2009}, 
MPS~\cite{voelter_language_2011},
Hazel~\cite{omar_hazelnut_2017}, 
WebFlow~\cite{noauthor_design_2014},
and others.
%%% Make it clear this is something our community should care about
Structure editing makes (syntactically or semantically) invalid programs impossible to represent, which promises to simplify the editing process for programmers at every skill level.

Existing collaboration techniques underlying tools such at Git or Live Share are text based.
Text editing involves linear edits, e.g., line or character insertions and deletions.
Structure editing, in contrast, involves tree edits, e.g., node insertions, deletions, and repositionings.
It is not clear how to generalize collaborative editing techniques rooted in text editing to a structured setting.
Our goal is to develop collaborative techniques that are rooted in structure editing.

% (Cyrus: Why is this a hard problem? All our existing tools are text based and it isn't clear how to do that correctly in a structured setting.)
In this paper, we introduce Grove, a \emph{collaborative structure editor calculus} based on Hazelnut~\cite{omar_hazelnut_2017}, a single-user structure editor calculus.
% (Introduce Grove: what makes it collaborative? what interesting thingG
In Grove, edits are commutative, i.e., they can be applied in any order, so collaboration is simply replay.
The shared edit state is represented as a graph, which is necessary to model interesting states that arise naturally in a collaborative setting, 
including direct conflicts and positioning conflicts. 
Users do not work with this graph directly---instead, the graph is decomposed into a collection of syntax trees with references between them, which we call a \emph{grove}.
The Grove calculus interprets a user's action on a grove as a set of graph edits which can be applied locally and sent to another user for replay.

% The rest of the paper is organized as follows.
% We begin in Sec.~\ref{sec:alice-and-bob} with example scenarios that demonstrate collaborative editing in Grove.
% Sec.~\ref{sec:formalism} formalizes the concepts introduced in the examples.
% Sec.~\ref{sec:related-work} describes related work.
% Sec.~\ref{sec:conclusion} concludes.

%%%%%%%%%%%%%%%%%%%%%%%%%%%%%%%%%%%%%%%%%%%%%%%%%%%%%%%%%%%%%%%%%%%%%%%%%%%%%%%%

\section{Grove by Example}
\label{sec:alice-and-bob}

% TODO: look for places to talk about the graph

% starting
Suppose Alice and Bob want to collaboratively edit a program using Grove.
For simplicity, we restrict ourselves to simple arithmetic expressions.
Alice and Bob collaborate by sending each other their edit actions as graph edits.
We assume nothing about the communication infrastructure.

% sketch and share 
Alice and Bob open their respective editors to a new shared program,
which they see as a cursor on an empty hole ($\hole$).
Alice begins by constructing a number (1).
The new expression fills the hole under the cursor.
Bob sees the hole until he receives Alice's edit,
at which point he sees the number.

% wrapping
Bob builds on Alice's work by constructing a plus.
Structurally, this requires ``wrapping'' a plus constructor around the existing expression, resulting in $\ePlus{}{1}{\hole}$.
Alice continues to see only the number until she receives Bob's edit,
at which point Alice also sees $\ePlus{}{1}{\hole}$.

% conflict
% TODO: Call out how this leads to a different result than if we replayed user actions instead of graph actions, i.e., emphasize the difference between text-based and structural edit conflicts. Make it clear that what's being said is "I want the left position of the + to be x or 2," as opposed to "I want to insert the character x or 2 at the given cursor position."

% Alice           Bob            
% 1 + >[]<        1 + >[]<         A: construct times; move to other child
% 1 + >[]< * []   1 + >[]<         B: construct x
% 1 + >[]< * []   1 + >x<          {sync: 
%                                     A: construct x 
%                                     B: construct times } 
% 1 + >x< * []    1 + >x< * []
% TODO: maybe allude to this kind of scenario in the related work

% Alice           Bob              
% 1 + >[]<        1 + >[]<         A: construct times
% 1 + >[]< * []   1 + >[]<         B: construct x
% 1 + >[]< * []   1 + >x<          {sync: 
%                                     A: construct x 
%                                     B: construct times } 
% 1 + >x< * []    1 + >x< * []

% Alice              Bob            
% 1 +^a >[]<         1 +^a >[]<       A: make R of a into times
% 1 +^a >[]< *^b []  1 +^a >[]<       B: make R of a into x
% 1 +^a >[]< *^b []  1 +^a >x<        {sync: 
%                                        A: make R of a into x
%                                        B: make R of a into times}

Both Alice and Bob move their cursors to the empty hole on the right.
Unfortunately, they have not yet agreed on what the next edit will be.
Alice constructs a times ($\eTimes{}{\hole}{\hole}$) whereas Bob constructs a variable ($x$).
They each send their edits to one another.
Because these edits occured concurrently, i.e., without any global relative ordering between them, we now have a conflict in the right position of the plus: is it a variable or a times?
This fact is explicitly represented as a conflict hole ($\ePlus{}{1}{\conflictHole{x, \eTimes{}{\hole}{\hole}}}$).

Conflict holes indicate where conflicts occur and which expressions are involved.
In contrast, naively applying the Hazel calculus (coincidentally and silently) merges the conflicting expressions if neither Alice or Bob moves their cursor during replay, 
and causes their edit states to diverge otherwise.
Applying a text-based technique would likely merge the characters `x' and `*'.
Because Grove interprets user actions as graph edits, replay is not affected by local cursor movements.

% deletion / detach
% TODO: get rid of this, but keep the detail that Bob's edits aren't lost.
% Bob moves his cursor to the variable and renames it to $y$.
% Before Alice receives Bob's edit,
% she decides to resolve the conflict by wrapping Bob's variable in the product.
% First, Alice ``deletes'' the variable by detaching it from the sum.
% Then, she relocates the detached variable to the left operand of the product.
% Alice now sees the new product in place of the conflict ($\ePlus{}{1}{\eTimes{}{x}{\hole}}$).
% However, Bob's edit to the variable is not lost.
% After the remaining edits are shared,
% both Alice and Bob see the new product and the renamed variable ($\ePlus{}{1}{\eTimes{}{y}{\hole}}$).

% restoration / multiparents
Alice resolves the conflict by moving her cursor to the conflict hole and deleting it.
Bob receives her edits, resulting in $\ePlus{}{1}{\hole}$.
Alice moves the number 1 to the right.
Bob does the same and then moves it back.
They send their edits to each other.
Now, one expression (the number) inhabits two positions, so we have a positioning conflict:
does it go on the left or the right?
This fact is represented with two multiparent references,
resulting in $\ePlus{}{\multiVertex{\e_1}}{\multiVertex{\e_2}}$.

In the graph, edges have identities, so multiple edges may have a common origin and destination.
Multiparent references allow us to model many-to-one relationships as trees.
References are symbolic links between expressions, each represented in the graph as an edge between vertices corresponding to the linked expressions.
In our current scenario, $\e_1$ and $\e_2$ are the edges that link the operands of the plus operator to the shared expression (the number 1).

% unicycles
% Alice and Bob decide to try a different approach.
% Alice deletes the references,
% moves her cursor onto the plus,
% and constructs a lambda which wraps the plus in its body.
% Bob receives Alice's edits, resulting in $\eFun{}{\hole}{\hole}{(\hole + \hole)}$.
% Bob moves his cursor to the right of the plus and constructs a times.
% Alice receives Bob's edits, resulting in $\eFun{}{\hole}{\hole}{(\hole + \hole * \hole)}$.

% Alice moves the left times operator into the right one and sees ($\ePlus{}{\hole}{\eTimes{}{\eTimes{}{\hole}{\hole}}{\hole}}$).
% Bob moves the right times operator into the left one and sees ($\ePlus{}{\eTimes{}{\hole}{\eTimes{}{\hole}{\hole}}}{\hole}$).
% They share the edits with each other and both of the times operators disappear!

% What happened?
% From the perspective of the graph, each term has a corresponding vertex.
% Moving a term corresponds to repositioning a vertex,
% which involves deleting its incoming edges and reconstructing them somewhere else.
% Although Alice and Bob separately deleted just the edge anchoring their respective times to the plus,
% the combined edits detached both times operators and attached them to each other to
% form a unicycle~\cite{kruskal_efficient_1990}:
% a chain of interdependent terms---%
% ($\eTimes{}{\cycleVertex{\e_2}}{\hole}$) and ($\eTimes{}{\cycleVertex{\e_3}}{\hole}$)%
% ---each containing a symbolic reference to the next link in the chain.

% Actions in the calculus are commutative, so all users will converge on the same state as edits are synchronized.

%%%%%%%%%%%%%%%%%%%%%%%%%%%%%%%%%%%%%%%%%%%%%%%%%%%%%%%%%%%%%%%%%%%%%%%%%%%%%%%%

\section{Formalism}
\label{sec:formalism}

We will now make the intuitions developed in the previous section more precise by defining the key concepts of our collaborative structure editor calculus.

% \begin{figure}
%   \[
%     \arraycolsep=0pt
%     \begin{array}{lcllll}
%       % t & {}\in{} & Term & {}::={} &
%       %   e
%       %   \mid \tau
%       %   \mid q
%       % \\
%       e & {}\in{} & Exp & {}::={} &
%         \eVar{G}{x}
%         \mid \eFun{G}{q}{\tau}{e}
%         \mid \eApp{G}{e}{e}
%         \mid \eNum{G}{n}
%         \mid \ePlus{G}{e}{e}
%         \mid \eTimes{G}{e}{e}
%         \mid \multiVertex{\e}
%         \mid \cycleVertex{\e}
%         \mid \conflictHole{e_i}_{i \leq n}
%         \mid \emptyHole{v}{p}
%       \\
%       % \tau & {}\in{} & Typ & {}::={} &
%       %   \tArrow{G}{\tau}{\tau}
%       %   \mid \tNum{G}
%       %   \mid \multiVertex{\e}
%       %   \mid \cycleVertex{\e}
%       %   \mid \conflictHole{\tau_i}_{i \leq n}
%       %   \mid \emptyHole{v}{p}
%       % \\
%       % q & {}\in{} & Pat{} & {}::={} &
%       %   \pVar{G}{x}
%       %   \mid \multiVertex{\e}
%       %   \mid \cycleVertex{\e}
%       %   \mid \conflictHole{q_i}_{i \leq n}
%       %   \mid \emptyHole{v}{p}
%       % \\
%     \end{array}
%   \]  
%   \caption{
%     The syntax of expressions.
%     Note that for conflict holes, $\protect\conflictHole{e_i}_{i \leq n}$,
%     it must be the case that $n \geq 2$.
%   }
%   \label{fig:syntax}
% \end{figure}

\subsection{Editor States}

A \emph{graph} $G : \E \to \Sigma$ is a function from edges to edge states.
An \emph{edge} $\e{=}(u, v, p, v') \in \E$ is a directed link originating from source vertex $v$ at position $p$ and targeting vertex $v'$.
An \emph{edge state} $s \in \Sigma{=}\SetOf{\bot, \Plus, \Minus}$ indicates the presence and liveness of an edge in the graph.
When $G(\e) = \Plus$, we say $\e$ is live.
When $G(\e) = \Minus$, we say $\e$ is deleted.
Otherwise, we say there is no liveness indicator for $\e$ in $G$, or that $\e$ is not in $G$.
The total ordering $\bot \sqsubset \Plus \sqsubset \Minus$ forms a lattice over $\Sigma$.

A \emph{vertex} $v{=}(u, k)$ is a uniquely identifiable instance of constructor $k$.
% Each vertex except a distinguished root $v_{root}$ corresponds to a term constructor,
% and each constructor has a sort: $Exp$, $Pat$, or $Typ$.
A \emph{position} is a named port from which edges may originate.
% Note that the only valid position from $v_{root}$ is the distinguished root position $p_{root}$.
% For example, in section~\ref{sec:alice-and-bob}, the initial edit state is an empty graph.
% When Alice constructs the number (1),
% a new vertex $v_1{=}(u_1, \ExpNum(1))$ is constructed
% and a new live edge $\e_2{=}(u_2, v_{root}, p_{root}, v_1)$ is added to $G$.
% Later, when Alice replaces the number with another term, $\e_2$ is deleted and so $G(\e_2) = \Minus$.
%
The \emph{arity} of a constructor relates each valid position with a corresponding sort.
In this document, we have only a single sort, $Exp$.
% For example, a lambda ($\eFun{}{q}{\tau}{e}$) has three positions%
% ---$\LamParam$, $\LamType$, $\LamBody$---%
% and its arity is \[\SetOf{(\LamParam, Pat), (\LamType, Typ), (\LamBody, Exp)}.\]
For an edge $\e{=}(u, (u_s, k_s), p, (u_t, k_t))$, we say $\e$ is \emph{well sorted} when the arity of $k_s$ relates $p$ to the sort of $k_t$.
When all of the edges in $G$ are well sorted, we say $G$ is well sorted.

Each well sorted graph can be decomposed into an equivalent collection of valid expressions,
with cycles broken by symbolic references,
and recomposed back into a graph.
The following grammar defines the set of valid expressions:
\[
  \arraycolsep=0pt
  \begin{array}{lcllll}
    % t & {}\in{} & Term & {}::={} &
    %   e
    %   \mid \tau
    %   \mid q
    % \\
    e & {}\in{} & Exp & {}::={} &
      \eVar{G}{x}
      \mid \eFun{G}{q}{\tau}{e}
      \mid \eApp{G}{e}{e}
      \mid \eNum{G}{n}
      \mid \ePlus{G}{e}{e}
      \mid \eTimes{G}{e}{e}
      \mid \multiVertex{\e}
      \mid \cycleVertex{\e}
      \mid \conflictHole{e_i}_{i \leq n}
      \mid \emptyHole{v}{p}
    \\
    % \tau & {}\in{} & Typ & {}::={} &
    %   \tArrow{G}{\tau}{\tau}
    %   \mid \tNum{G}
    %   \mid \multiVertex{\e}
    %   \mid \cycleVertex{\e}
    %   \mid \conflictHole{\tau_i}_{i \leq n}
    %   \mid \emptyHole{v}{p}
    % \\
    % q & {}\in{} & Pat{} & {}::={} &
    %   \pVar{G}{x}
    %   \mid \multiVertex{\e}
    %   \mid \cycleVertex{\e}
    %   \mid \conflictHole{q_i}_{i \leq n}
    %   \mid \emptyHole{v}{p}
    % \\
  \end{array}
\]
A \emph{grove} $\gamma = (\Theta_{NP}, \Theta_{MP}, \Theta_{U})$ is a triple of sets of such expressions, denoted $\Theta$, each corresponding to vertices that respectively have no parents, have multiple parents, or are unicycle roots.
When $\gamma$ can be recomposed into a well sorted graph, we say $\gamma$ is well sorted.

The core of our formal semantics consists of two functions: $\decompOp$ and $\recompOp$.
The former transforms graphs into groves and
the latter transforms groves into graphs.
The following theorem establishes an isomorphism between well sorted graphs and well sorted groves.
%
\begin{theorem}
  If $G$ is a well sorted graph,
  then there exists a well sorted grove $\gamma$
  such that $\decomp{G} = \gamma$
  and $\recomp{\gamma} = G$.
\end{theorem}

% Introduce the graph. Give an example connecting to Sec. 2.
% Introduce terms and groves.
% Outline how they're connected: what decomp / recomp are (don't overdo it)
% (Inverse relationship)

\subsection{Edits}

A \emph{graph edit} $\alpha \in \SetOf{\Plus, \Minus} \times \E$ pairs an edge state with an edge.
We specify the semantics of graph actions with the transition relation
\[
  \applyAction{G}{(s,\e)}{\extends{G}{\e \mapsto s \sqcup G(\e)}}
\]
where $s_1 \sqcup s_2$ is the least upper bound of edge states $s_1$ and $s_2$ according to the $\sqsubset$ ordering.
The following theorem establishes that graph edits are commutative.
%
\begin{theorem}
  \label{thm:commutativity}
  Let $G, G'$ be well sorted graphs,
  and let $\alpha_1, \alpha_2$ be graph edits.
  If $\applyAction{G}{\alpha_1}{\applyAction{G_1}{\alpha_2}{G'}}$ for some $G_1$,
  then $\applyAction{G}{\alpha_2}{\applyAction{G_2}{\alpha_1}{G'}}$ for some $G_2$.
\end{theorem}
%
Consequently,
if we assume that users eventually stop editing and all edits are eventually shared,
then graphs are a form of CRDT~\cite{shapiro_conflict-free_2011} with respect to graph edits.

Although our semantics are centered on graph edits, users do not edit the graph directly.
% Instead, they edit the corresponding grove.
A \emph{user action} is a high level operation that the formal semantics interpret as one or more graph edits.
The Grove calculus supports three cursor-relative user actions on expressions: Construct, Delete, and Reposition.
For example, when Alice moves the numer 1 from left to right, she places her cursor on the number 1 and issues a Reposition action with the new destination.
The formal semantics interpret the action with the following rule,
\begin{mathpar}
  \inferrule[Reposition]{
    \edges{G_e} = \SetOf{\e_i}_{i \leq n} \\
    (p, \sort{k_{G_e}}) \in \arity{k}
  }{
    \applyAction{
      \cursor{e}
    }{\Reposition{v{=}(u, k)}{p}}{
      \SequenceOf{\graphEdit{\Minus}{\e_i}}_{i \leq n};
      \graphEdit{\Plus}{(\fresh{u}, v, p, v_{G_e})}
    }
  }
\end{mathpar}
%
which produces two graph edits: $(\Minus, \e_3); (\Plus, \e_4)$,
where $\e_3$ is the old edge from the left position of the plus
and $\e_4$ is a new edge from the right position.
%
Here, $v_{G_e}{=}(u_e, k_{G_e})$ is the vertex corresponding to $e$,
and $G_e$ is the subgraph of $G$ of all incoming edges to $v_{G_e}$,
and $\cursor{~}$ denotes the position of the cursor,
and $\fresh{u}$ is a fresh identifier.

% What the graph edits are.
% State the commutativity theorem (converging editor states when users communicate)
% Point out that commutative calculus ==> this is a CRDT
% User actions are higher level, but map onto graph edits: give an example
% Show judgment form "user action --> graph edits" and give an example.

% Results: commutativity theorem
% Contribution: clean collaborative editing (via the commutativity theorem)

% (Next steps (maybe): looking forward to semantic structure editing)

%%%%%%%%%%%%%%%%%%%%%%%%%%%%%%%%%%%%%%%%%%%%%%%%%%%%%%%%%%%%%%%%%%%%%%%%%%%%%%%%

\section{Related Work}
\label{sec:related-work}

% Strategy:
% - "near" (conceptually or technically) work that helps understand what Grove is / does.
% - competing or contrasting work
% 
% Purpose:
% - establish novelty
% - entry points into related history

% Collaborative editors allow multiple users to work concurrently on a shared edit state.
% In combination, this feature set is unique to Grove.

% Grove is the first CRDT-based collaborative structure editor calculus.

The Grove calculus is an adaptation of Hazelnut~\cite{omar_hazelnut_2017} for collaborative editing.
In Scratch~\cite{resnick_scratch_2009}, a well known structure editor for children,
programs are visually composed of blocks corresponding to AST nodes.
In structure editing calculi such as Hazelnut and Grove,
valid updates must satisfy a sensibility constraint to maintain valid syntax.

CRDTs~\cite{shapiro_conflict-free_2011} are data types with commutative updates that guarantee all replicas converge to the same state without explicit synchronization techniques such as mutual exclusion.
Existing work focuses largely on collaborative text editing~\cite{sun_real_2020},
or on a narrow application domain, such as
JSON authoring~\cite{kleppmann_conflict-free_2017},
co-CAD~\cite{lv_novel_2018}, 
and structuring large documents~\cite{hall_causal_2018}.
CRDTs for graphs have been limited to adaptations to simpler structures, e.g., as sets~\cite{shapiro_convergent_2011} or trees~\cite{hall_causal_2018}.
To our knowledge, Grove is the first structure editing calculus that uses a graph CRDT for collaboration.

Operational transformation is another approach to real-time collaborative editing.
It imposes a causal ordering on updates, e.g., with Lamport clocks~\cite{lamport_time_nodate}.
Because peers must communicate to enforce causality,
operational transformations are ostensibly less well suited than CRDTs for applications on unreliable networks.
However, recent work has shown operational transformations are more popular in practice~\cite{sun_real_2020}.

The diffTree algorithm~\cite{mcclurg_difftree_nodate} is similar to Grove, but based on operational transformation.
The diffTree algorithm's conflict resolution algorithm is, however,
asymmetric and therefore not strongly eventually consistent,
so peers must coordinate to resolve conflicts.

% formal version control systems?

%%%%%%%%%%%%%%%%%%%%%%%%%%%%%%%%%%%%%%%%%%%%%%%%%%%%%%%%%%%%%%%%%%%%%%%%%%%%%%%%

\section{Conclusion}
\label{sec:conclusion}

This paper introduced Grove, a collaborative structure editor based on a novel graph CRDT data type.
By reimagining collaborative text editing around communicating structural updates,
as opposed to communicating line- or character-level text edits,
Grove brings the kind of functionality found in collaborative text editors to the structured setting.

% Connect the results with collaborative structure editing: we (can now / should be able to) get the same kind of functionality as collaborative text editing in a structured setting. 
% This paper reimagines some of it around communicating structure edits, as opposed to communicating line- or character-level text edits.
% We leave this for future work.
% convey any excitement about the future here, not before

%%
%% The next two lines define the bibliography style to be used, and
%% the bibliography file.
\bibliographystyle{ACM-Reference-Format}
\bibliography{Grove}

\end{document}